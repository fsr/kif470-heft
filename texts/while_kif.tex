\addchap{Externes Rahmenprogramm}
Zeitgleich zu unserer KIF ist in Dresden eine Menge los. Dieses Kapitel gibt einen kurzen Abriss der Veranstaltungen, falls ihr uns untreu werden wollt.

\subsection*{Schöne neue Cyberwelt}
Anlässlich der Feierlichkeiten zu 50 Jahre Informatik in Dresden wird eine Ausstellung zum Thema Videospiele im Ratssaal des Fakultätsgebäudes Stattfinden.
Die Aufbauten dafür beginnen bereits während der KIF und für KIFfel besteht die Möglichkeit, am Samstag zwischen 10 und 16 Uhr die dort vorbereiteten Exponate zu besichtigen.
Der große Ratssaal befindet sich direkt neben dem Orgabüro im 1. Stock des Andreas-Pfitzmann-Baus.


\subsection*{BRN -- Bunte Republik Neustadt}
Die Bunte Republik Neustadt ist am besten als Stadtteilfest der Neustadt zu beschreiben.
Einwohnern der Neustadt zufolge ist die BRN allerdings gar kein Fest sondern ein Dauerzustand.
Die Feierlichkeiten vom 14. bis 16. Juni sind lediglich eine Jubiläumsveranstaltung.
Für diesen Zeitraum verwandelt sich der Stadtteil in eine große Partyzone.
Überall gibt es Stände mit Essen, Musik und Straßenkünstler sind unterwegs.
Es bot sich an, die Kneipentour parallel zu legen.
Das Programm der BRN findet ihr hier \link{http://brn-programm.de/}.

\subsection*{LNdW -- Lange Nacht der Wissenschaften}
"Wissenschaft statt Kissenschlacht!"
Die Lange Nacht der Wissenschaften findet am 14. Juni von 18 bis 1 Uhr statt.
An diesem Abend kann man verschiedene wissenschaftliche Einrichtungen besuchen.
Es gibt diverse Vorträge, Auftritte, man kann Labore besuchen und es gibt viel auszuprobieren.
Das Programm ist sehr vielfältig und unter \link{http://www.wissenschaftsnacht-dresden.de/programm/} zu finden.

\subsection*{Rammstein}
Am 12. und 13. Juni wird das Rudolf-Harbig-Stadion in Dresden beben.
An diesen beiden Tagen findet dort das Rammsteinkonzert (Europe Stadium Tour 2019) statt.
Der Einlass zur Veranstaltung beginnt dort ab 19 Uhr.

\begin{awesomeblock}[KIFteal]{2pt}{\faQuestion}{KIFteal}
    \textbf{Adresse des Rudolf-Harbig-Stadions:}

    Lennestraße 12

    01069 Dresden
\end{awesomeblock}

% TODO: xkcd

\vfill

\begin{awesomeblock}[KIFgreen]{2pt}{\includegraphics[width=.1\textwidth]{img/fledermaus.pdf}}{KIFgreen}
\begin{minipage}[t]{.82\textwidth}
  \footnotesize\textbf{Fun Fact:}

  Warum wir Fledermäuse auf dem Cover haben?\\

  Das ist die in Dresden sehr bekannte und beliebte kleine Hufeisennase.
  Diese Fledermaus ist an der Waldschlößchenbrücke beheimatet, durch deren Bau Dresdens Elbuferwiesen ihren Weltkulturerbetitel verloren haben. Sie ist so selten, dass kaum ein Dresdner sie je gesehen hat. Trotzdem war sie der Grund, weswegen der schwer umstrittene Bau der Waldschlößchenbrücke fast gescheitert wäre.
  Letztlich wurde die Brücke trotzdem gebaut, aber stark auf die Bedürfnisse der kleinen Tiere angepasst.
  Aus diesem Grund herrscht zur Flugzeit der Tierchen immer ein Tempolimit von 30 km/h auf der Brücke. Das hat übrigens noch einmal 220.000\,€ extra gekostet.
\end{minipage}
\end{awesomeblock}
