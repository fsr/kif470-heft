\addchap{Links}

Alle Links sind auch direkt als \url{kif.ifsr.de/<Zahl>} aufrufbar.

{%
\small
\begin{longtable}{r p{11cm}}
\linklist%
\end{longtable}
}

\vfill

\begin{awesomeblock}[KIFgreen]{2pt}{\includegraphics[width=.1\textwidth]{img/fledermaus.pdf}}{KIFgreen}
\begin{minipage}[t]{.82\textwidth}
  \footnotesize\textbf{Fun Fact:}

  Warum wir Fledermäuse auf dem Cover haben?\\

  Das ist die in Dresden sehr bekannte und beliebte kleine Hufeisennase.
  Diese Fledermaus ist an der Waldschlößchenbrücke beheimatet, durch deren Bau Dresdens Elbuferwiesen ihren Weltkulturerbetitel verloren haben. Sie ist so selten, dass kaum ein Dresdner sie je gesehen hat. Trotzdem war sie der Grund, weswegen der schwer umstrittene Bau der Waldschlößchenbrücke fast gescheitert wäre.
  Letztlich wurde die Brücke trotzdem gebaut, aber stark auf die Bedürfnisse der kleinen Tiere angepasst.
  Aus diesem Grund herrscht zur Flugzeit der Tierchen immer ein Tempolimit von 30 km/h auf der Brücke. Das hat übrigens noch einmal 220.000\,€ extra gekostet.
\end{minipage}
\end{awesomeblock}
