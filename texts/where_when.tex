\addchap{\texorpdfstring{Where \& When}{Where and When}}

\section*{QR Codes}
Für den Fall, dass ihr euch des Weges von A nach B einmal nicht sicher sein, oder ihr ein Faible für digitale Raumpläne und Karten haben solltet kommt ihr mit den folgenden QR Codes auf die Website des Campusnavigators.

\section*{Die Fakultät -- APB}
Der Andreas-Pfitzman Bau, Heimat der Fakultät Informatik und euer Haupt-Aufenthaltsort während der KIF\@.
Bitte beachtet, dass im gesamten Gebäude (wie auch in allen anderen Campusgebäuden) Rauchverbot herrscht.
Den Rest der Hausordnung fällt in die Kategorie \enquote{gesunder Menschenverstand}, sollte also niemanden einschränken.

Im inneren Außenbereich könnt ihr am Teich entspannen und das hoffentlich schöne Wetter genießen.
Beachtet aber, dass der Teich ein Lebensraum für Fische und ähnliche Lebewesen mit Kiemen, also keine Menschen ist.

\section*{Der Parkplatz}
Die Fläche vor der  Fakultät wird für die KIF zweckentfremdet und euch verschiedenstes bieten.  So wird am Donnerstagabend das KIF Open Air auf dem Parkplatz stattfinden. Dafür haben wir für euch folgende Bands organisiert:
Aber auch sonst lohnt es sich immer mal wieder vorbeizuschauen, wenn euch der Sinn nach Spiel und Spaß steht.

\section*{KIF-Café (E023)}
Im einzigen Hörsaal des APB und größtem Raum des Foyers findet ihr das KIF-Café.
Dafür haben wir mit den Traditionen der letzten KIFs gebrochen und die Einteilung in mehrere Räume über Bord geworfen.
Als zentraler Treffpunkt und Aufenthaltsraum steht das Café für Spiel und Spaß bereit, aber auch zur Entspannung und für die Pausen zwischen den AKs.
Für genügend Steckdosen zum Laden von Handy und Laptop wurde gesorgt.

\section*{Ewiges Frühstück und AK Tee (E008/E007)}
Gegenüber des Cafés findet ihr das Ewige Frühstück, KIF-typisch wortwörtlich ewig.
Hier könnt ihr euch rund-um-die-Uhr am Essen bedienen.
Getränketechnisch gibt es die Auswahl zwischen mehreren Säften und natürlich Wasser.
Wem das nicht ausreicht, der kann im Raum nebenan auf eine Tasse Tee entspannen -- Sofas inbegriffen.
Gerade im Raum der AK Tee herrscht wie der Name schon sagt, ein einem Arbeitskreis gerechter Geräuschpegel.

\section*{Kasse des Vertrauens – \texttt{ascii} (E016)}
Zwar gibt es bei dieser KIF nur ein offizielles KIF-Café, aber die Räumlichkeiten des \texttt{ascii}, unseres Fakultätscafé, stehen uns glücklicherweise auch zur Verfügung.
Dort findet ihr die Kasse des Vertrauens (KDV).
Hier können Erfrischungsgetränke, auch alkoholischer Natur, Süßigkeiten und kleinere Snacks erworben und in klassischem Café-Ambiente genossen werden.

\section*{Infopoint (E017)}
Solltet ihr einmal Probleme/Fragen haben, scheut euch nicht den Infopunkt, passend im Büro unseres FSR neben dem \texttt{ascii} angesiedelt, aufzusuchen.
Hier werden wir euch unterstützen, sofern die Fragen nicht schon in diesem Heft oder mit einer einfachen Google-Suche beantwortet sind.

\section*{PC-Pools (E065, E067, E069)}
Neben der normalen Seminarräume, bieten wir euch auch Zugang zu der Hälfte der PC-Pools der Fakultät, hauptsächlich angedacht für AKs die PCs benötigen, aber auch für die persönliche (Uni-) Arbeit.

\section*{Die Turnhalle}
Direkt neben der Fakultät, bei Blick auf die Fakultät zur Linken, liegt die Turnhalle.
Dort findet ihr eure Schlafplätze und die Duschen für die eigene Hygiene.
Hier wird es vermutlich genügend Platz zur Aufbewahrung eures Gepäcks geben.
Ist dies nicht der Fall, so werden wir über Ausweichpläne informieren.
Generell ist der Verzehr von alkohol- und zuckerhaltigen Getränken in den Turnhallen nicht gestattet.
Wer auf der sicheren Seite schlafen möchte, nimmt maximal eine Flasche wasser mit zum schlafen.

\section*{Der Barkhausen--Bau}
Auf dem Weg zwischen Alter Mensa und APB findet ihr den Barkhausen--Bau.
Hier werden unter Umständen AKs stattfinden, sollte im APB nicht mehr genügend Platz verfügbar sein.
Um einen Besuch dieses schönen im Stil der \enquote{neu--sachlichen Formensprache} errichteten Baus werdet ihr davon unabhängig wahrscheinlich nicht herumkommen, so ist der Barkhausen--Bau Heimat des:

\section*{Plena -- Heinz Schönfeld Hörsaal}
Das Hauptevent der Konferenz, die Rede ist von Abschlussplenum.
Dieses findet wie auch schon das Anfangsplenum im denkmalgeschütztem brandneuen Heinz Schönfeld Hörsaal statt.
Da dieser ein wenig schwieriger zu finden ist, lohnt sich hier unter Umständen der Campusnavigator.

\section*{Essen -- Wo und was?}
Neben des ewigen Frühstücks (und etwaigen Attraktionen auf dem Parkplatz…), werden wir donnerstags am Teich grillen, sodass ihr euch mit reichlich Fleisch, Käse und Gemüse den Magen vollstopfen könnt.
Außerdem werden vom Infopoint Waffeleisen in Umlauf gebracht werden, mit denen ihr euch selbst und viel wichtiger andere Kiffel nach Lust und Laune versorgen könnt.

Abgesehen davon hat der Campus auch mensatechnisch einiges zu bieten, so habt ihr die Wahl zwischen vier Mensen um euren Hunger zu stillen.
Die Speisepläne findet ihr online unter (Link).
Bezahlen könnt ihr mithilfe der Mensakarte die ihr bei der Anmeldung erhalten haben solltet und so das Essen zu den normalen studentischen Preisen genießen.
Von Haus aus sind diese bereits mit 20 Euro geladen -- locker genug für das tägliche Mittagessen.
Prinzipiell könnt ihr euch das Geld natürlich frei einteilen und euch zum Beispiel auch durch alle Nachtische durchprobieren.
Ist eure Karte leer, könnt ihr Sie an der Kasse des Vertrauens, oder an der Kasse der Mensen wieder aufladen.
Solltet ihr die Karte verloren, oder andere Probleme haben, helfen wir euch am Infopoint gerne weiter.

\subsection*{Alte Mensa}
Ca. 5 Minuten Fußweg vom APB aus und ihr erreicht die Alte Mensa, die größte und schönste Mensa der TU\@.
Da sie die nächste Mensa von der Fakultät aus ist und gleichzeitig auch die meisten Gerichte anbietet, dürfte die selbsternannte \enquote{kulinarische Schlagader im Herzen des Campus} eigentlich alle eure Bedürfnisse decken.
Der Vollständigkeit halber noch eine kurze Auflistung der alternativen Mensen.

\subsection*{Siedepunkt}
Am anderen Ende des Campus, ca. 20 Minuten zu Fuß von der Fakultät entfernt, liegt der Siede.
Den relativ weiten Weg, machen die langen Öffnungszeiten wett/alternativlos, so ist der Siedepunkt die einzige Mensa mit Abendangebot in der vorlesungsfreien Woche.

\subsection*{U-Boot}
Die Bio-Mensa auf dem Campus setzt sich mit der stilvollen Einrichtung von den anderen Mensen ab, leider auch durch die deutliche höheren Preise, aufgrund der teureren Zutaten aus biologischer Erzeugung.
Fußweg ca. 15 Minuten.

\subsection*{Zeltschlösschen}
Klassische Mensa, die der Alten Mensa in so ziemlich allen Punkten unterlegen ist, könnt ihr den weiteren Weg in Kauf nehmen, falls euch das Angebot der Alten Mensa nicht zusagt.

Neben den Mensen können wir euch das Firat, den Dönerladen mit perfekter Lage auf dem Campus wärmstens empfehlen.
Neben der Qualität die wohl allen Studenten der TU bekannt sein dürfte, locken die gemütliche Einrichtung und die vergleichsweise große Auswahl an Gerichten.
