\addchap{So Krams der schon interessant zu wissen ist}

In diesem Kapitel findet ihr alle wichtigen Informationen rund um die \KIF{} in Dresden in komprimierter Form.
Man hätte dieses Kapitel also durchaus auch mit ``Organisatorisches'' überschreiben können, aber wäre kein so schöner Eye-Catcher.

\section*{Orga-Ansprechpartner}

Falls es mal irgendwo klemmen sollte oder etwas nicht funktioniert, wendet euch am Besten zuerst an die Ansprechpartner der KIF-Orga.
Diese residiert im Orga-Büro im kleinen Ratszimmer (\textbf{APB/1005}), welches rund um die Uhr besetzt ist.
Alternativ ist sie per Mail unter \textbf{kif-orga@ifsr.de} oder per Telefon unter \textbf{+49 162 1234567} erreichbar.
Falls ihr mit einem bestimmten Mitglied der Orga Kontakt aufnehmen wollt, könnt ihr dies über folgende Kanäle tun:

\vspace{.5cm}

\begin{tabular}{l l l}
Manuel Thieme:      & 0351 463 319127685 & manuel@ifsr.de \\[.05cm]
Matthias Stuhlbein: & 0351 463 319127685 & stuhlbein@ifsr.de \\[.05cm]
Alice:              & 0178 128 189301155 & pizza@ifsr.de \\[.05cm]
Pius Meinert:       & 7892 082789345     & foobar@ifsr.de
\end{tabular}

\section*{Siehst du die Farben?}

% TODO: Wenn T-Shirt-Farben feststehen, hier die farbigen Dots anpassen
Wie auf jeder KIF geben auch hier Farben die jeweilige Statusgruppe an, zu der ein KIFfel gehört.
Ein {\,\color{ShirtAttendee}$\bullet$\,}petrolfarbenes T-Shirt, wie du es vermutlich gerade in deinem KIF-Beutel gefunden hast, weist dich als Mitglied der Statusgruppe \emph{Teilnehmer} aus.
Solltest du im Verlaufe der KIF in höhere Gefilde aufsteigen und dich an der Durführung beteiligen wollen, erhältst du in deiner Funktion als \emph{Engel} ein {\,\color{ShirtAngel}$\bullet$\,}grünes T-Shirt.
\emph{Orga}-Mitglieder sind leicht an ihrem {\,\color{ShirtOrga}$\bullet$\,}roten Shirt zu erkennen.

\section*{Ist das der Sonderzug nach Pankow?}

\begin{wrapfigure}{r}{0.2\textwidth}
  \vspace{-12pt}
  \begin{centering}
    \includegraphics[width=0.2\textwidth,keepaspectratio]{img/dvb_mobil.png}
  \end{centering}
  \vspace{-15pt}
\end{wrapfigure}

Wenn ihr während eures Aufenthaltes in Dresden von A nach B kommen wollt, ohne auf Fahrraddiebstahl und Fußwege zurückgreifen zu müssen, können wir euch den öffentlichen Personennahverkehr wärmstens empfehlen.
Auf der mobilen Webseite der DVB (siehe QR-Code) könnt ihr euch über mögliche Verbindungen informieren.
Vom APB und den Turnhallen aus gelangt ihr am Besten mit der Tram-Linie 3 vom Münchner Platz aus in die Innenstadt.

\section*{Nicht-Sanktionierter Zugang zum Neuland}

In's Internet gelangt ihr auch an unserer Uni über das WiFi-Netzwerk \textbf{eduroam} mit eurem universitätsspezifischen Login.
Sollte eure Uni nicht am Eduroam-Programm teilnehmen oder ihr aus einem anderen Grund keine Logins besitzen, meldet euch im Orga-Büro.
Dort können wir euch mit Gast-Accounts für die Zeit eures Aufenthaltes versorgen.
Falls etwas nicht funktioniert, könnt ihr auch auf das unverschlüsselte Netz VPN/Web zurückgreifen.

% TODO: Change this heading!
\section*{Ohne Netz und auf dem Trockenen}

Die Turnhallen bieten euch nachts einen Platz zum schlafen und Freunden der regelmäßigen Seifenbenutzung eine ausreichende Anzahl Duschen.
Allerdings gelten für die Hallen selbst eine Reihe von Restriktionen, um eure Sicherheit und die Nutzbarkeit der Turnhallen für ein Turnier am Sonntag zu gewährleisten.
Aus Brandschutzgründen stehen euch keine Steckdosen in den Hallen zur Verfügung, damit im Falle des Falles niemand über Kabel fällt.
Außerdem dürfen zum Schutz des Hallenbodens \textbf{keine Getränke} in der Halle verzehrt werden.
Wasser stellt in diesem Zusammenhang eine Ausnahme dar, da dieses eine reinigende Wirkung hat.

\section*{For the Power-Users}

Strom für eure Notebooks und Handys könnt ihr aus den gekennzeichneten Steckdosen im KIF Café und in den AK-Räumen beziehen.
Außerdem gibt es eine dedizierte Ladestation für eure Geräte, an der ihr eure Geräte gern auch über Nacht -- wann immer das für euch gerade ist -- laden könnt.
In den Turnhallen selbst stehen aus Brandschutzgründen (herumliegende Kabel etc.) keine Steckdosen zur Verfügung.

\section*{Smoke on the Wodka}

Bitte beachtet, dass Rauchen ausschließlich in den Außenbereichen bei den Aschenbechern gestattet ist.
Generell ist der Verzehr von alkohol- und zuckerhaltigen Getränken in den Turnhallen nicht gestattet.
Wer auf der sicheren Seite schlafen möchte, nimmt maximal eine Flasche wasser mit zum schlafen.
