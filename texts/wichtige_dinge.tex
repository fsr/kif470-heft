\addchap{So Krams der schon interessant zu wissen ist}

In diesem Kapitel findet ihr alle wichtigen Informationen rund um die \KIF{} in Dresden in komprimierter Form.
Man hätte dieses Kapitel also durchaus auch mit ``Organisatorisches'' überschreiben können, aber das wäre kein so schöner Eye-Catcher.

\section*{Info-Point. Da werden Sie geholfen!}

Falls es mal irgendwo klemmen sollte oder etwas nicht funktioniert, gibt es eine wohldefinierte Eskalationshierarchie, über die ihr eure Fragen propagieren könnt.

\begin{awesomeblock}[KIFteal]{2pt}{\faQuestion}{KIFteal}
    \textbf{Eskalationshierarchie bei Fragen:}

  Altkiffel $\longrightarrow$ Infopoint $\longrightarrow$ KIF-Orga
\end{awesomeblock}

Zuerst einmal solltet ihr euch immer an Altkiffel wenden.
Durch ihre nahezu unbegrenzte Weisheit und Kampferfahrung sind sie meist in der Lage, euch direkt zu helfen.
Wenn du selbst ein Altkiffel bist oder dir von keinem geholfen werden konnte, ist der nächste Schritt der Info-Point im \textbf{APB/E017}.
% TODO: Ist der immer offen?
Der letzte Schritt sollte dann die KIF-Orga sein.
Diese residiert im Orga-Büro im kleinen Ratszimmer (\textbf{APB/1005}), welches rund um die Uhr besetzt ist.
Alternativ ist sie per Mail unter \textbf{kif-orga@ifsr.de} oder per Telefon unter \textbf{+49 162 1234567} und \textbf{+49 351 463 ?????} erreichbar.

\section*{Siehst du die Farben?}

Wie auf jeder KIF geben auch hier Farben die jeweilige Statusgruppe an, zu der ein KIFfel gehört.
Ein {\,\color{ShirtAttendee}$\bullet$\,}petrolfarbenes T-Shirt, wie du es vermutlich gerade in deinem KIF-Beutel gefunden hast, weist dich als Mitglied der Statusgruppe \emph{Teilnehmer} aus.
Solltest du im Verlauf der KIF in höhere Gefilde aufsteigen und dich an der Durführung beteiligen wollen, erhältst du in deiner Funktion als \emph{Engel} ein {\,\color{ShirtAngel}$\bullet$\,}grünes T-Shirt.
\emph{Orga}-Mitglieder sind leicht an ihrem {\,\color{ShirtOrga}$\bullet$\,}roten Shirt zu erkennen.

\section*{Ist das der Sonderzug nach Pankow?}

\begin{wrapfigure}[6]{r}{0.2\textwidth}
  \vspace*{-11pt}
  \textcolor{KIFgrey}{\qrcode[height = .2\textwidth]{https://m.dvb.de}}
\end{wrapfigure}

Wenn ihr während eures Aufenthaltes in Dresden von A nach B kommen wollt, ohne auf Fahrraddiebstahl und Fußwege zurückgreifen zu müssen, können wir euch den öffentlichen Personennahverkehr wärmstens empfehlen.
Auf der mobilen Webseite der DVB \link{https://m.dbd.de/} (siehe auch QR-Code) könnt ihr euch über mögliche Verbindungen informieren.
Vom APB und den Turnhallen aus gelangt ihr am Besten mit der Tram-Linie 3 vom Münchner Platz aus in die Innenstadt.

\section*{Nicht-Sanktionierter Zugang zum Neuland}

Ins Internet gelangt ihr auch an unserer Uni über das WiFi-Netzwerk \textbf{eduroam} mit eurem universitätsspezifischen Login.
Sollte eure Uni nicht am Eduroam-Programm teilnehmen oder ihr aus einem anderen Grund keine Logins besitzen, meldet euch am \emph{Info-Point}.
Dort können wir euch mit Gast-Accounts für die Zeit eures Aufenthaltes versorgen.
Falls etwas nicht funktioniert, könnt ihr auch auf das unverschlüsselte Netz VPN/Web zurückgreifen.

\section*{Matratzensport mit Hindernissen}

Die Turnhallen bieten euch einen Platz zum Schlafen und Freunden der regelmäßigen Seifenbenutzung eine ausreichende Anzahl an Duschen.
Da die Hallen die zeitlichen Restriktionen des ewigen Frühstücks folgen, könnt ihr hier \textbf{egal zu welcher Uhrzeit} eure Ruhe finden.
Allerdings gelten für die Hallen selbst eine Reihe von Einschränkungen, um eure Sicherheit und die Nutzbarkeit der Turnhallen für ein Turnier am Sonntag zu gewährleisten.
Aus Brandschutzgründen stehen euch \textbf{keine Steckdosen} in den Hallen zur Verfügung, damit im Falle des Falles niemand über Kabel fällt.
Außerdem dürfen zum Schutz des Hallenbodens \textbf{keine Getränke} in der Halle verzehrt werden.
Wasser stellt in diesem Zusammenhang eine Ausnahme dar, da dieses eine reinigende Wirkung hat.

\section*{For the Power-Users}

Strom für eure Notebooks und Handys könnt ihr aus den gekennzeichneten Steckdosen im KIF-Café und in den AK-Räumen beziehen.
% TODO: Wo ist die???
Außerdem gibt es eine dedizierte Ladestation für eure Geräte \textbf{im KIF-Café}, an der ihr eure Geräte gern auch über Nacht -- wann immer das für euch gerade ist -- laden könnt.
In den Turnhallen selbst stehen aus Brandschutzgründen (herumliegende Kabel etc.) keine Steckdosen zur Verfügung.

\section*{Smoke on the Wodka}

Bitte beachtet, dass Rauchen ausschließlich in den Außenbereichen bei den Aschenbechern gestattet ist.
Auch der Konsum sonstiger Drogen sollte nur im Rahmen gesetzlicher Bestimmungen stattfinden. \textbf{Und auf keinen Fall in der Halle.}
