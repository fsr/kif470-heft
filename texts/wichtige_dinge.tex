\addchap{So Krams der schon interessant zu wissen ist}

In diesem Kapitel findet ihr alle wichtigen Informationen rund um die KIF 47,0 in Dresden in komprimierter Form.
Man hätte dieses Kapitel also durchaus auch mit ``Organisatorisches'' überschreiben können, aber wäre kein so schöner Eye-Catcher.

\section*{Orga-Ansprechpartner}

Falls es mal irgendwo klemmen sollte oder etwas nicht funktioniert, wendet euch am Besten zuerst an die Ansprechpartner der KIF-Orga.
Diese residiert im Orga-Büro im kleinen Ratszimmer (\textbf{APB/1005}), welches rund um die Uhr besetzt ist.
Alternativ ist sie per Mail unter \textbf{kif-orga@ifsr.de} oder per Telefon unter \textbf{+49 162 1234567} erreichbar.
Falls ihr mit einem bestimmten Mitglied der Orga Kontakt aufnehmen wollt, könnt ihr dies über folgende Kanäle tun:

\vspace{.5cm}

\begin{tabular}{l l l}
Manuel Thieme:      & 0351 463 319127685 & manuel@ifsr.de \\[.05cm]
Matthias Stuhlbein: & 0351 463 319127685 & stuhlbein@ifsr.de \\[.05cm]
Alice:              & 0178 128 189301155 & pizza@ifsr.de \\[.05cm]
Pius Meinert:       & 7892 082789345     & foobar@ifsr.de
\end{tabular}

\section*{Siehst du die Farben?}

% TODO: Wenn T-Shirt-Farben feststehen, hier die farbigen Dots anpassen
Wie auf jeder KIF geben auch hier Farben die jeweilige Statusgruppe an, zu der ein:e Teilnehmer:in gehört.
Ein {\,\color{KIFteal}$\bullet$\,}orangenes T-Shirt, wie du es vermutlich gerade in deinem KIF-Beutel gefunden hast, weist dich als Mitglied der Statusgruppe \emph{Teilnehmer} aus.
Solltest du im Verlaufe der KIF in höhere Gefilde aufsteigen und dich an der Durführung beteiligen wollen, erhältst du in deiner Funktion als \emph{Engel} ein {\,\color{KIFteal}$\bullet$\,}petrolfarbenes T-Shirt.
\emph{Orga}-Mitglieder sind leicht an ihrem {\,\color{KIFteal}$\bullet$\,}roten Shirt zu erkennen.

\section*{Ist das der Sonderzug nach Pankow?}

\begin{wrapfigure}{r}{0.2\textwidth}
  \vspace{-12pt}
  \begin{centering}
    \includegraphics[width=0.2\textwidth,keepaspectratio]{img/dvb_mobil.png}
  \end{centering}
  \vspace{-15pt}
\end{wrapfigure}

Wenn ihr während eures Aufenthaltes in Dresden von A nach B kommen wollt, ohne auf Fahrraddiebstahl und Fußwege zurückgreifen zu müssen, können wir euch den öffentlichen Personennahverkehr wärmstens empfehlen.
Auf der mobilen Webseite der DVB (siehe QR-Code) könnt ihr euch über mögliche Verbindungen informieren.
Vom APB und den Turnhallen aus gelangt ihr am Besten mit der Tram-Linie 3 vom Münchner Platz aus in die Innenstadt.

\section*{Nicht-Sanktionierter Zugang zum Neuland}

In's Internet gelangt ihr auch an unserer Uni über das WiFi-Netzwerk \textbf{eduroam} mit eurem universitätsspezifischen Login.
Sollte eure Uni nicht am Eduroam-Programm teilnehmen oder ihr aus einem anderen Grund keine Logins besitzen, meldet euch im Orga-Büro.
Dort können wir euch mit Gast-Accounts für die Zeit eures Aufenthaltes versorgen.
Falls etwas nicht funktioniert, könnt ihr auch auf das unverschlüsselte Netz VPN/Web zurückgreifen.

\section*{For the Power-Users}

% TODO: Wie gekennzeichnet? o.O
Strom für eure Notebooks und Handys könnt ihr aus den gekennzeichneten Steckdosen beziehen.

\section*{Smoke on the Wodka}

% TODO: Rauchen und Alkohol Policy?

% TODO: Kam das hier?
% \subsection*{Mensa Mensa mjam mjam mjam}
