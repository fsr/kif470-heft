\addchap{Hackerbrausenvorträge}

Wir haben für euch an den drei vollen KIF-Tagen drei Vorträge über bekannte und nicht bekannte
Hackergetränke organisiert. Diese finden von Donnerstag bis Samstag jeweils von 16:00 bis 18:00 Uhr
statt. Natürlich haben wir jeweils die Unternehmensführer eingeladen, die nur darauf warten, von
euch mit Fragen überhäuft zu werden. Die vorgestellten Getränke sind zu großen Teilen in der Kasse
des Vertrauens zu erwerben. Prost.

\section*{Reineckes Getränke -- Donnerstag 16:00 Uhr}
Als kleiner Getränkehändler aus Hamburg wollten Sie nur eines: Mate in 0,33l Flaschen. Da dies von
den marktführenden Marken nicht zu ermöglichen war, haben Ludger und Hans-Werner einfach ihre eigene
Mate gebraut. Heute ist die \textbf{Flora-Power} von keinem Hackerevent mehr wegzudenken. Aus der
47.0. KIF gibt es sie natürlich auch. Mehr informationen und eine genauere Geschichte des
Unternehmens und seiner Getränke gibt es bei diesem Vortrag.

\section*{zickzack -- Freitag 16:00 Uhr}
Die Lokale Mate in Dresden ist die \textbf{kolle-mate}. Gerüchten zufolge wurden die ersten Chargen
dieses köstlichen Getränkes in einer Badewanne gebraut. Das Bemerkenswerte an der zickzack GmbH ist
die unternehmensführung als Kollektiv, von der in diesem Vortrag einiges Berichtet werden kann.

\section*{Baikal Getränke -- Samstag 16:00 Uhr}
Vielleicht wird euch bei der Getränkeauswahl, die es in der Kasse des Vertrauens zu erwerben gibt,
die Vielzahl an exotischen Geschmacksrichtungen und -kombinationen der \textbf{WOSTOK} limonaden
auffallen. Birne-Rosmarin, Pflaume-Kardamom und Tannenwald sind nur einige von ihnen. Wie man auf
die Idee kommt, diese Zutaten in Getränken zu verbauen, wird euch Joris in seinem Vortrag erzählen.
Bis Samstag solltet ihr genügend Zeit haben, zumindest einige der Wostoks auszuprobieren.
