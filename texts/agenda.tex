\addchap{Agenda 2019}

Jede KIF ist per default durch den wechselnden Austragungsort immer etwas besonderes.
Wir haben uns für die SommerKIF aber zusätzliche Besonderheiten überlegt.
Diese müssen kein Alleinstellungsmerkmal bleiben.
Wir freuen uns über jede Fortführung gelungener Programmpunkte auf zukünftigen KIFs.
Doch nicht alle arrangierten Köstlichkeiten haben wir in diesem Heft beschrieben; Wo bliebe denn da die Überraschung?

\section*{Fachvortrag}
Prof. Bernhard Ganter ist Professor der Mathematik im Bereich der algebraischen Strukturen und hat lange Jahre den Informatikern die Einführungsveranstaltung in die Mathematik gelesen.
Diejenigen die ihn bis 2013 in dieser Vorlesung erleben durften sind nachhaltig von seiner Einzigartigkeit überzeugt.
Für die \KIF{} reist er nach Dresden und wird einen Fachvortrag zum Thema \enquote{Wir können auch anders --- Mathematik für Informatiker}.
Es handelt sich also um einen informatischen Ausflug in die Mathematik.

\section*{K:O:A -- Das KIF Open-Air}
Wer am Donnerstag in Richtung Parkplatz schaut, wird den Aufbau von Bühne und Beschallungsanlage beobachten können.
Ab 18:00 Uhr wird es dort 4 Stunden lang auf die Ohren geben.
Die organisierten Künstler sind eine feine Auswahl an lokalen Musikern.
\vspace{10pt}

\begin{tabular}{rl}
    18:00 & \textbf{Fede} -- gut gelaunter Pop-Rock \\

    19:20 & To be announced\\

    20:40 & \textbf{DJ STE} -- Chiptunes, Bitpop, 8-Bit and Game Music!
\end{tabular}

\section*{Hackerbrausenvorträge}
Wir haben für euch an den drei vollen KIF-Tagen drei Vorträge über bekannte und nicht bekannte Hackergetränke organisiert.
Diese finden von Donnerstag bis Samstag jeweils von 16:00 bis 18:00 Uhr statt.
Natürlich haben wir jeweils die Unternehmensführer eingeladen, die nur darauf warten, von euch mit Fragen überhäuft zu werden.
Die vorgestellten Getränke sind zu großen Teilen in der Kasse des Vertrauens zu erwerben. Prost.

\subsection*{Reineckes Getränke -- Donnerstag 16:00 Uhr}
Als kleiner Getränkehändler aus Hamburg wollten Sie nur eines: Mate in 0,33l Flaschen.
Da dies von den marktführenden Marken nicht zu ermöglichen war, haben Ludger und Hans-Werner einfach ihre eigene Mate gebraut.
Heute ist die \textbf{Flora-Power} von keinem Hackerevent mehr wegzudenken.
Auf der 47.0. KIF gibt es sie natürlich auch.

\subsection*{zickzack -- Freitag 16:00 Uhr}
Die lokale Mate in Dresden ist die \textbf{kolle-mate}.
Gerüchten zufolge wurden die ersten Chargen dieses köstlichen Getränkes in einer Badewanne gebraut.
Das Bemerkenswerte an der zickzack GmbH ist die Unternehmensführung als Kollektiv.

\subsection*{Baikal Getränke -- Samstag 16:00 Uhr}
Vielleicht wird euch bei der Getränkeauswahl, die es in der Kasse des Vertrauens zu erwerben gibt, die Vielzahl an exotischen Geschmacksrichtungen und -kombinationen der \textbf{WOSTOK} Limonaden auffallen.
Birne-Rosmarin, Pflaume-Kardamom und Tannenwald sind nur einige von ihnen.
Bis Samstag solltet ihr genügend Zeit haben, zumindest einige der Wostoks auszuprobieren.

\section*{Komplexität der Unisextoiletten}
Unisextoiletten sind ein muss auf jeder KIF. Auch die Unterscheidung zwischen pissoirierten und nicht pissoirierten Räumlichkeiten zur Verrichtung von Notdürften ist nichts neues.
Wir haben allerdings unsere kreativsten Köpfe zum Rauchen gebracht und die Bezeichnungen auf P und NP geändert.
Nun kann jeder von euch selbst einmal überprüfen, ob P = NP gilt oder nicht.

\section*{Wasserspende}
Es gibt auf dieser KIF zwei verschiedene Wasserspendeklassen, die an ihren Beschriftungen entsprechend zu erkennen sind: Leitungswasserspende und Mineralquellwasserspende.
Für zweiteres ist ein nicht vernachlässigbarer Mehraufwand in die Produktion geflossen während ersteres von uns selbst gezapft wurde.
Habt ihr eure eigene Flasche vergessen oder besitzt gar keine solche, lässt sich für diesen Anwendungsfall auch eine Mateflasche sehr gut missbrauchen.

\section*{Notfallromantik}
Zu einer exzellenten Universität gehören exzellente Gebäude mit einer exzellenten Gebäudesteuerung (EGS).
Die EGS des APB der TUD ist SMART und aus diesem Grund sind all ihre Entscheidungen ohne Fragen zur Kenntnis zu nehmen.
Dazu gehört auch das regelmäßige Ausschalten der Beleuchtung in sämtlichen Räumen des Gebäudes.
Doch keine Angst, in den sanitären Einrichtungen wurden für solche Fälle Notfallbeleuchtungen angebracht.
Die dadurch hervorgebrachte Notfallromantik sollte entweder ignoriert oder entsprechend genutzt werden.
Nachforschungen haben ergeben, dass die KI hinter der EGS das Fehlen von Liebe als Beleidigung auffasst.
Bisher konnte allerdings noch kein unmittelbarer Zusammenhang zur Beleuchtungssteuerung nachgewiesen werden.
Wir bitten alle Vorkommnisse und Erkenntnisse an @mstuhlbein zu melden.

\section*{This KIF has a very green color}
Wir haben versucht, viele Punkte der KIF-Durchführung nachhaltig zu gestalten.
Das führt allerdings dazu, dass auf die ein oder andere Gewohnheit verzichtet werden musste.
Bei Anregungen zu mehr Nachhaltigkeit ist die KIF-Orga mit Eintritt des Gedankenganges unmittelbar zu informieren.

\subsection*{Abweichungen in der Shirtfarbe}
Die Teilnehmershirts, die ihr zur Anmeldung erhalten habt, sind zum großen Teil bio, nachhaltig und fair.
Leider gibt es durch die Einschränkung auf diese Attribute keine Option, mit einem Produkt die gesamte für die KIF notwendige Bandbreite an Größen abzudecken.
Größen $\geq$ 3XL sind daher nicht explizit bio, nachhaltig und fair und können in der Farbe abweichen.

\subsection*{Diversity of lanyards}
Im Zuge der Bestrebungen die einer KIF beiwohnende Materialschlacht auf ein nachhaltiges Minimum zu reduzieren, haben wir den Frankfurter BYOL Ansatz fortgeführt, um nur die tatsächlich benötigte Menge an Lanyards zu organisieren.
Dadurch kann es vorkommen, dass ihr Menschen gegenübersteht, die ein viel bunteres, schöneres, längeres, dickeres oder gar kein Band mit ihrem Badge führen.
Sprecht diese Personen doch einfach mal auf die Geschichte ihres Schlüsselbands an.
Vielleicht entwickelt sich dadurch eine lange Freundschaft; Oder ein grausamer Mord.

\subsection*{Unsterblichkeit durch fehlende Beschichtung}
Der Weg weg vom Kunststoff führte uns zur Entscheidung, das Papier eurer Badges dicker zu wählen, direkt zu lochen.
Außerdem haben wir uns gegen beschichtetes Papier ausgesprochen und somit das Problem erzeugt, dass es nicht länger möglich ist, Mörderspielpunkte verlustfrei zu entfernen.
Dieses Problem konnte in Vorbereitung auf die KIF nicht gelöst werden.
Wir dotieren daher die erste Lösung dieses Problems mit einer Flasche Mate.
